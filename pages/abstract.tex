\thispagestyle{empty}

\section*{Abstract}
Generative artificial intelligence is reshaping software engineering practices by enhancing and automating core tasks, including code generation, debugging and program repair. Despite these advancements, existing automated program repair approaches suffer from complexity, high computational demands, and a lack of integration within practical software development lifecycles.
\\
In this thesis, we address these challenges by introducing a novel and lightweight Automated Bug Fixing system leveraging Large Language Models (LLMs), explicitly designed for seamless integration into continuous integration pipelines. Our containerized solution automates the bug-fixing lifecycle, from GitHub issue creation to the generation and validation of pull requests, reducing manual intervention and streamlining development workflows.\\
We evaluated our approach using the QuixBugs benchmark, testing twelve LLMs for effectiveness, efficiency, and costs. The results demonstrate repair success rates of up to 100\% with short execution times and low costs, highlighting the practicality of our streamlined solution in effectively repairing small-scale software bugs. 
\\
The outcomes underscore the feasibility and potentials of integrating APR directly into continuous integration pipelines. Nevertheless, limitations such as reliability concerns and potential security vulnerabilities remain. Enhancements like adaptive model selection and improved integration techniques are outlined to further refine and optimize this promising approach in the future. 
