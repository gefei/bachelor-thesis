Using the results of the evaluation and 

\section{Validity}

- Quixbugs a small dataset, not representative of real world software development
- only python, not representative of real world software development
- but shows the potential of applying llm based agents in a real world CD/CD environment
External QuixBugs only, internal non-deterministic LLM, construct tests is not complete oracle

\section{Potentials}
- can take over small tasks in encapsulated environment without intervention
- small models can solve more problems with retrying with feedback
- no description is needed to solve small issues
- this concept is applicable to other python repositories
- configuration makes it adjustable to different repositories and environments
- similar results to other approaches -> is feasible

- combines agentless and a bit of interactive but interactivity is limited due to timings but can resemble real world remote environment

-with small models and attempt loop makes small models pass the whole benchmark?

agent architectures produce good results epically paired with containerized environments. \cite{puvvadiCodingAgentsComprehensive2025}

\ref{houLargeLanguageModels2024}
- accelerate bug fixing
- lets developers focus on more complex tasks
- therefor enhance software reliability and maintainability

\section{Limitations}
- github actions from github have a lot of computational noise
- workflow runs on every issue and therefor has some ci minute overhead this could be solved by using a github app which replies on webhook events
-- SECURITY ISSUE: Prompt injection in issue: CI/CD makes this a bit safer?
- its limited to small issues

\section{Summary of Findings}

\section{Lessons Learned}
- ai is a fast moving field with a lot of noise


\section{Roadmap for Extensions}
- Service Accounts for better and more transparent integration
- try out complex agent architectures and compare metrics and results
- try out more complex bug fixing tasks - SWE bench
- concurrency and parallelization of tasks