\section{Background and Motivation}
Generative AI is changing the software industry and how software is developed. The emergence of Large Language Models (LLMs) has opened up new possibilities for automating various aspects of software engineering, including code generation, debugging, and program repair. These models have demonstrated remarkable capabilities in understanding and generating code snippets, making them valuable tools for developers.
Since debugging and fixing bugs are critical tasks in software development, the integration of LLMs into Automated Program Repair (APR) systems has gained significant attention. Recent studies have shown that LLMs can effectively assist in identifying and fixing bugs, thereby improving the efficiency of the software development process.
However, the integration of these practices into existing software development workflows remains understudied. 

exsitng apporaches are often complex and require significant computational resources, making them less suitable for budget-constrained environments. Additionally, the lack of integration with Continuous Integration and Continuous Deployment (CI/CD) practices limits their practical applicability in real-world scenarios.
The motivation behind this thesis is to explore the potential of LLMs in enhancing automated program repair (APR) systems, particularly in continuous integration and continuous deployment (CI/CD) environments. By leveraging the capabilities of LLMs, we aim to develop a more efficient and cost-effective approach to automated bug fixing that can seamlessly integrate into existing software development workflows.



\section{Problem Statement}
modern APR systems requiremt a lot of computational power budget and manual effort - stduies suggest emphasizing connections with DevOps Procedures \cite{puvvadiCodingAgentsComprehensive2025}

\section{Objectives and Research Questions}
The primary objective of this thesis is to investigate the feasibility and effectiveness of LLMs in APR within the Software Development Lifecycle (CI/CD pipelines) in budget restrained environment. The research questions guiding this investigation include: