Generative AI is rapidly changing the software industry and how software is developed and maintained. The emergence of Large Language Models (LLMs), a subfield of Generative AI, has opened up new opportunities for enhancing and automating various domains of the software development lifecycle. Due to remarkable capabilities in understanding and generating code snippets, LLMs have become valuable tools for developers' everyday tasks such as requirement engineering, code generation, refactoring, and program repair \cite{houLargeLanguageModels2024, puvvadiCodingAgentsComprehensive2025}.
\\
Despite these advances, bug fixing remains a resource-intensive and often negatively perceived task \cite{winterHowDevelopersReally2023}, resulting in  interruptions and context switching that reduce developers' productivity \cite{vasilescuSkyNotLimit2016}
The process bug fixing can be time-consuming and error-prone, leading to delays in software delivery and increased costs \cite{}. In fact, acording to  CISQ: in 2022 alone 607 billion dollars were spend for finding an fixing bugs only the US\cite{CostPoorSoftware}.
\\
Given that debugging and fixing bugs are such critical tasks in software development Automated Program Repair (APR) systems have gained significant attention.
Typically, bug fixing involves multiple steps: bug reporting, localization, repair, and validation \cite{zhangEmpiricalStudyFactors2012, leeUnifiedDebuggingApproach2024,xiaAgentlessDemystifyingLLMbased2024,zhangPATCHEmpoweringLarge2025, wangEmpiricalResearchUtilizing2025}.
Recent research has shown that LLMs can effectively be used for bug localization and repair, thereby setting new standards in the APR world and improving the efficiency of the software development process \cite{xiaAgentlessDemystifyingLLMbased2024,liuMarsCodeAgentAInative2024,yangSWEagentAgentComputerInterfaces2024, sobaniaAnalysisAutomaticBug2023, xiaAutomatedProgramRepair2024, huCanGPTO1Kill2024}.
However existing APR apporaches are often complex and require significant computational resources \cite{rondonEvaluatingAgentbasedProgram2025, }, making them less suitable for budget-constrained environments or solo developers. Additionally, the lack of integration with exisitng software development workflows/lifecycles limits their practical applicability in real-world envrionments \cite{chenUnveilingPitfallsUnderstanding2025,liuMarsCodeAgentAInative2024}.
\\
Motivated by these challenges, this thesis explores the potential of integrating LLM based automated bug fixing within continuous integration and continuous deployment (CI/CD) pipelines. By leveraging the capabilities of LLMs, we aim to develop a cost-effective prototype for automated bug fixing that seamlessly integrates into existing software development workflows. Considering computational depands, complexity of integration and practical contraints we aim to provide insights into possibilities and limitations of out approach.


% while writing this paper lots of reasearch and tools have been published clearly showing the importance of the topic and the need for further research in this area.


