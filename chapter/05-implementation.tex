In this section we break down the implementation of the system into its core components, following the methodology and requirements outlined in the previous sections. The full code is attached in the \ref{Appendix} appendix.

The goal was to create a system which not only fixes bugs but is also portable /deployable across different repositories and configurable to some extend.

The resulting system consists of two main components. The \textbf{APR core} which hold the core logic for the repair process and lives inside a docker image, making it portable and easy to deploy. The second component is the \textbf{CI/CD Pipeline} which integrates the core logic with the GitHub repository. It serves as the entry point and orchestrates the execution of the APR core based on issues and configured triggers in the repository.

To fulfill requirement X some of the systems behavior can be controlled by adjusting a configuration. A configuration is optional when no configuration is in place the system will use default configuration which is defined in the code. The configuration is written in YAML and can be placed at the root of the repository. This allows for easy customization of the system without changing the code itself. The configuration file is named 'bugfix.yml' and is read by the APR core and CI Pipeline during execution. The configuration allows for controlling labels, workdir, branches, attempts and LLM models used for the repair process. 
\begin{table}[ht]
    \centering
    \small
    \renewcommand{\arraystretch}{1.5}
    \begin{tabular*}{\textwidth}{@{\extracolsep{\fill}} p{5cm} | p{4cm} @{}}
        \toprule
        \textbf{Configuration Field} & \textbf{Description} \\
        
        \midrule
        \bottomrule
    \end{tabular*}
    \caption{Overview of APR Benchmarks}
    \label{table:configuration}
\end{table}



\section{System Components}
\textbf{Agent Core:}\\
The agent core is written in python and dockerized so its slim and portable.
the agent core contains the main bug fixing logic.
To start fixing bugs it needs the following environment:
The repo where to fix files on - provided using docker volume mount
the following Environment Variables:
- Github Token
- Github Repo
- LLM API key
- ISSUE TO PROCESS
- Github repository

With this environment set the system loops over all issues which are fetched from the environment variables.

For each issue the main APR logic is executed. This main logic consists of tools and stages:
---IMAGE here
First the workspace (a new branch based on configuration naming) and a repair context is set up. the context is hing of the program its needed at every step and works as the main data structure for the APR system like memory.
--- JSON of Context init here:
A stage uses the context to perform a specific task in the bug fixing process and returns the context with its added context. The stages are: Localize, Fix, Build, Test


With a prepared workspace the repair process start with the localization stage. This stage construct a prompt for the LLM to localize the bug in the codebase. This prompt makes use of a constructed hierarchy of the repositories file structure and the issue description. The LLM is expected to return a list of files and lines where the bug is located.
--- PROMPT

The results are stored in the context.

With the the localized files in the context the fix stages constructs a prompt with the file content and the issue description. the llm can return code or no changes needed.
--- Prompt
these edits are then applied to the files in the workspace. The context is updated with the new file content.

To ensure the changes are properly formatted the the build stages formats the code using the black formatter and lints the python code to ensure maintainable code.

Next up the test stage runs tests for the fixes files and attaches these results to the context.

if tests do not pass the system will record a new attempt and start over from the state of the fix stage. Additionally a feedback is generated using the previous attempts code and results from the context which gets added to the fix prompt.

if the validation passes or  the maximum number of attempts is reached the system will either report and unsuccessful repair to the issue or continue to the application steps

on successful repair the following steps are executed:
the file changes are committed and a diff file is generated.
the changes are pushed to the remote repo using the github token

a pull request is opened with a description of the changes and a link to the issue and more details about tests and some metrics like the number of attempts and the tokens used for the repair process.

During execution the core logs its actions, which can be used for debugging and analysis. Furthermore it collects metrics such as the number of attempts, execution time, and token usage, which are essential for evaluating the performance of the APR system. Logs and metrics are saved as .log and json files at the end

The agent core is designed to be modular and extensible, allowing for future enhancements and additional stages or tools to be integrated as needed. It is also designed to be lightweight, ensuring that it can run efficiently within the constraints of a CI/CD environment.



\textbf{CI/CD Pipeline:}\\
The Pipeline is written in YAML acoording to the Github CICD standard. \ref{}
We will use runners hosted by Github which takes away the overhead of managing our own runners but comes at the cost of unknown performance and availability
It is made up of the Triggers and 3 Jobs: 'gate', 'skipped' and 'bugfix'.
The triggers are set up first and will allow the execution of the APR process.
Triggers can results in two types of runs:
processing of all bug issues in correct state on manual execution request of the workflow (``workflow\_dispatch'') or scheduled execution (``cron'').
processing a single issue: when an issue is opened and label with the configured labels (``issue\_opened and issue\_labeled '') or when extra information is added or edited on an issue in form of a comment.

The trigger  event information gets passed as environment variables to the next job
``gate'' which is responsible for evaluating if the issue should be processed or skipped. This job checks the labels and resolves the issue state to determine if the issue is relevant for the APR process.
If no issues pass this gate the job ``skipped'' is executed, which simply logs that no issues were found to process and exits the workflow.

When the gate outputs issues that should be processed the job ``bugfix'' is executed. This job checks out the current repositiory and mounts it as a volume to the agent core container. Addionally it sets the necessary environment variables mentioned at \ref{}. For the agent to work perimmsions are set on the job level to allow the agent to edit repository content, create pull requests and write issues.

For giving access to the agent cores logs and metrics the job provides the logs directory as an artifact which is available after the workflow run is completed.


for all of this to run the following environment secrets need to be defined in the repo:


\section{System Architecture}

IMAGE of Figma diagram



\section{System Configuration}




secrets that need to be added:
LLM provider API key, GITHUB TOKEN
need to enable GitHub actions permission to create pull requests in repo settings

explain fields:

The full implementation is listed in Appendix~\ref{Appendix}