To guide the implementation of the prototype, we developed the following requirements to help design the system. These requirements enable better planning and prioritization during development and allow for tracking progress throughout the implementation. We constructed both functional \ref{tab:functional-requirements} and non-functional \ref{tab:non-functional-requirements} requirements.

\section{Functional Requirements}

\renewcommand{\arraystretch}{1.5} % Set row spacing to 1.5
\begin{longtable}{@{\extracolsep{\fill}} p{0.5cm} | p{3cm} | p{6cm} | p{4cm} @{}}
    \caption{Functional requirements} \label{tab:functional-requirements} \\
    \hline
    \textbf{ID} & \textbf{Title} & \textbf{Description} & \textbf{Verification} \\
    \hline
    \endfirsthead

    \hline
    \endfoot
        F0 \label{f0} & Multi Trigger
        & The Pipeline can be triggered: manually, scheduled via cron or by issue creation/labeling.
        & Runs can be found for these triggers \\ \hline
        F1 \label{f1} & Issue Gathering
        & Retrieve GitHub repository issues and filter them for correct state and configured labels \texttt{BUG}.
        & \texttt{gate} logs list of fetched issues.  \\ \hline
        F2 \label{f2} & Code Checkout
        & Fetch the repository code into a fresh workspace and branch (via Docker mount).
        & After F2, the Core has access the source files. \\ \hline
        F3 \label{f3} & Issue Localization
        & Use LLM to analyze the issue description and identify relevant files.
        & LLM output contains file paths with files that shall be edited. \\ \hline
        F4 \label{f4} & Fix Generation
        & Use LLM to edit the identified files.
        & LLM output contains adjusted content for the identified files. \\ \hline
        F5 \label{f5} & Change Validation
        & Run format, lint and relevant tests and capture pass/fail status.
        & Context shows build and test results. \\ \hline
        F6 \label{f6} & Iterative Patch Generation
        & If F5 reports failures, retry F4-F5 up to \texttt{max\_attempts} times.
        & Multiple stage execution are shown in context when Validation fails \\ \hline
        F7 \label{f7} & Patch Application
        & Commit LLM-generated edits to the issue branch.
        & Git shows a new branch with a commit referencing the Github issue \\ \hline
        F8 \label{f8} & Result Reporting
        & Report using Pull Requests and issue comments.
        & A PR or appears for each issue, showing diff and summary. \\ \hline
        F9 \label{f9} & Log and Metric Collection
        & Provide log files and Metrics for evaluation and debugging.
        & Log and metric files are accessible \\
\end{longtable}

\section{Non-Functional Requirements}

\renewcommand{\arraystretch}{1.5} % Set row spacing to 1.5
\begin{longtable}{@{\extracolsep{\fill}} p{0.5cm} | p{3cm} | p{6cm} | p{4cm} @{}}
    \caption{Non-Functional requirements} \label{tab:non-functional-requirements} \\

    \hline
    \textbf{ID} & \textbf{Title} & \textbf{Description} & \textbf{Verification} \\
    \hline
    \endfirsthead

    \hline
    \endfoot
        N1 \label{n0} & Containerized Execution
        & All APR code runs in CI runner in a Docker container.
        & Workflow shows Docker container usage \\ \hline
        N2 \label{n1} & Configurability
        & User can specify issue labels, branches, attempts, LLM models.
        & Changing the config file alters agent behavior accordingly. \\ \hline
        N3 \label{n2} & Portability
        & The system can be deployed on any python repository on GitHub.
        & The system repairs at least one issue in more than one repository \\ \hline
        N4 \label{n3} & Reproducibility
        & Runs are deterministic given identical repo state and config.
        & Multiple runs on the same issue report similar metrics. \\ \hline
        N5 \label{n4} & Observability
        & The system provides logs and metrics.
        & Logs and metrics files are generated for each run. \\ \hline
        N6 \label{n5} & No Manual Intervention
        & The system runs fully automatically without user input.
        & Issue creation to fix Pull Request works without any manual steps. \\ \hline
\end{longtable}
