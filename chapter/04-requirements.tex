To guide the implementation of the prototype, we developed the following requirements to help design and develop the prototype. These requirements enable structured planning and prioritization during development and track progress throughout the implementation. Both functional (Table\ref{tab:functional-requirements}) and non-functional (Table\ref{tab:non-functional-requirements}) requirements are listed below.

\section{Functional Requirements}

\renewcommand{\arraystretch}{1.5} % Set row spacing to 1.5
\begin{longtable}{@{\extracolsep{\fill}} p{0.5cm} | p{2.1cm} | p{6cm} | p{4.5cm} @{}}
    \caption{Functional requirements} \label{tab:functional-requirements} \\
    \hline
    \textbf{ID} & \textbf{Title} & \textbf{Description} & \textbf{Verification} \\
    \hline
    \endfirsthead

    \hline
    \endfoot
        F0 \label{f0} & Multiple \newline Triggers
        & The Pipeline can be triggered: manually, scheduled or by issue labeling in GitHub.
        & Runs can be found for \newline these triggers \\ \hline

        F1 \label{f1} & Issue \newline Filtering
        & The system retrieves and filters GitHub issues based on the \newline configured label.
        & Logs show list of passing \newline issues.  \\ \hline

        F2 \label{f2} & Code \newline Checkout
        & The pipeline fetches the repository code into a fresh workspace and branch.
        & Checkout and branch visible in logs.  \\ \hline

        F3 \label{f3} & Bug \newline Localization
        & The LLM analyzes the issue description and identifies the file(s) \newline needing changes.
        & \ac{LLM} output contains \newline file paths. \\ \hline

        F4 \label{f4} & Fix \newline Generation
        & The LLM generates and proposes code edits for the identified files.
        & \ac{LLM} output contains \newline adjusted content. \\ \hline

        F5 \label{f5} & Change \newline Validation
        & All generated changes are validated via formatting, linting, and running relevant tests.
        & Logs show results of \newline format/lint/test steps. \\ \hline

        F6 \label{f6} & Iterative \newline Patch \newline Generation
        & If F5 reports failures, retry F4-F5 until maximum number of attempts is reached.
        & Logs show retries with \newline multiple stage executions.  \\ \hline

        F7 \label{f7} & Patch \newline Application
        & Commit LLM-generated edits to the issue branch.
        & Git branch and commit with reference to the issue is visible. \\ \hline

        F8 \label{f8} & Result \newline Reporting
        & The system reports the outcome by creating a Pull Request or adding a comment to the issue.
        & Pull Requests and issue \newline comments are visible.\\ \hline

        F9 \label{f9} & Log and \newline  Metric \newline Collection
        & For every run, logs and key metrics are collected and made available.
        & Artifacts and console logs are available in CI run \newline  results. \\
\end{longtable}

\section{Non-Functional Requirements}

\renewcommand{\arraystretch}{1.5} % Set row spacing to 1.5
\begin{longtable}{@{\extracolsep{\fill}} p{0.5cm} | p{2.1cm} | p{6cm} | p{4.5cm}  @{}}
    \caption{Non-Functional requirements} \label{tab:non-functional-requirements} \\

    \hline
    \textbf{ID} & \textbf{Title} & \textbf{Description} & \textbf{Verification} \\
    \hline
    \endfirsthead

    \hline
    \endfoot
        N0 \label{n0} & Container \newline Runtime
        & All core logic runs inside a Docker container in the CI environment.
        & CI logs confirm use of \newline Docker container. \\ \hline

        N1 \label{n1} & Configurable
        & Users can configure labels, \newline branches, number of attempts and LLM in a YAML file.
        & The system loads a custom configuration. \\ \hline

        N2 \label{n2} & Portable
        & The system can be deployed and run on any Python GitHub repository with minimal setup.
        & Demonstrate successful \newline run on at least two \newline different repositories. \\ \hline

        N3 \label{n3} & Run \newline Repeatable
        & Runs are deterministic given identical repo state and config.
        & Multiple runs on the same issue report similar metrics. \\ \hline

        N4 \label{n4} & Observable
        & Logs and metrics are generated for every run and accessible as CI artifacts.
        & Downloadable artifacts \newline and visible logs for each run. \\ \hline

        N5 \label{n5} & End-to-End \newline Automation
        & The process is fully automated from issue creation to PR no manual intervention is needed.
        & No manual steps are \newline needed, automation is \newline shown. \\ \hline
\end{longtable}
