For development we developed the following requirements which guide the implementation and design of the prototype.
Following the INVEST model \cite{10.5555/984017} we constructed functional \ref{tab:functional-requirements} and nonfunctional \ref{tab:non-functional-requirements} requirements.

These requirements allow for better planning and prioritization during implementation and tracked progress.

\section{Functional Requirements}

\renewcommand{\arraystretch}{1.5} % Set row spacing to 1.5
\begin{longtable}{@{\extracolsep{\fill}} p{0.5cm} | p{3cm} | p{6cm} | p{4cm} @{}}
    \caption{Functional requirements} \label{tab:functional-requirements} \\

    \hline
    \textbf{ID} & \textbf{Title} & \textbf{Description} & \textbf{Verification} \\
    \hline
    \endfirsthead

    \hline
    \endfoot
        F0 \label{f0} & Multi Trigger
        & The Pipeline can be triggered: manually, scheduled via cron or by issue creation/labeling.
        & Runs can be found for these triggers \\ \hline
        F1 \label{f1} & Issue Gathering
        & Retrieve GitHub repository issues and filter them for correct state and configured labels \texttt{BUG}.
        & \texttt{gate} logs list of fetched issues.  \\ \hline
        F2 \label{f2} & Code Checkout
        & Fetch the repository code into a fresh workspace and branch (via Docker mount).
        & After F2, workspace/ contains the correct source files. \\\hline
        F3 \label{f3} & Issue Localization
        & Use LLM to analyze the issue description and identify relevant files.
        & LLM output contains file paths with files that shall be edited. \\\hline
        F4 \label{f4} & Fix Generation
        & Use LLM to edit the identified files.
        & LLM output contains adjusted content for the identified files. \\\hline
        F5 \label{f5} & Change Validation
        & Run format, lint and relevant tests and capture pass/fail status.
        & Logs/Context shows build and test results. \\\hline
        F6 \label{f6} & Iterative Patch Generation (retry logic)
        & If F5 reports failures, retry F4-F5 up to \texttt{max\_attempts} times.
        & After retries, either F4 passes or fails with no further retries. \\\hline
        F7 \label{f7} & Patch Application
        & Commit LLM-generated edits to the issue branch.
        & Git shows a new branch with a commit referencing the Github issue \\\hline
        F8 \label{f8} & Result Reporting
        & Open a PR or post a comment on GitHub with the diff.
        & A PR or appears for each issue, showing diff and summary. \\\hline
        F9 \label{f9} & Logs and Metrics Collection
        & Provide log files and Metrics with fix-rate, attempt history, timings, token usage.
        & A metrics file contains fields: issue, success, timings, stages. \\

        % F1 & Per Issue Workflow
        % & Execute F1--F4 for each fetched issue in sequence and aggregate results.
        % & For N issues, produce N PRs (or "no-fix" comments) and N metric entries. \\\hline
        % F12 & GitHub Integration
        % & Use the GitHub API (with \texttt{GITHUB\_TOKEN}) to list issues, create branches, open PRs, and post comments.
        % & All GitHub API calls succeed without manual intervention. \\ \hline
\end{longtable}

\section{Non-Functional Requirements}

\renewcommand{\arraystretch}{1.5} % Set row spacing to 1.5
\begin{longtable}{@{\extracolsep{\fill}} p{0.5cm} | p{3cm} | p{6cm} | p{4cm} @{}}
    \caption{Non-Functional requirements} \label{tab:non-functional-requirements} \\

    \hline
    \textbf{ID} & \textbf{Title} & \textbf{Description} & \textbf{Verification} \\
    \hline
    \endfirsthead

    \hline
    \endfoot
        N1 \label{n0} & Containerized Execution
        & All APR code runs in CI runner in a Docker container.
        & Workflow shows Docker container usage \\\hline
        N2 \label{n1} & Configurability
        & User can specify issue labels, branches, attempts, LLM models via YAML.
        & Changing the config file alters agent behavior accordingly. \\\hline
        N3 \label{n2} & Portability
        & The system can be deployable on any repository on GitHub.
        & ??? \\\hline
        N4 \label{n3} & Reproducibility
        & Runs are deterministic given identical repo state and config.
        & Multiple runs on the same issue report similar metrics. \\\hline
        N5 \label{n4} & Observability
        & The system provides logs and metrics.
        & Logs and metrics files are generated after each run. \\
\end{longtable}