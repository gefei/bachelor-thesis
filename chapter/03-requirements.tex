- for this prototype we constructed Requirements which the system shall statisfy
- we split into functional requirements: t.3.1 (EXPLAINATION), nonfucntional Requirements combined with security requriemnts

- these requiremtns allow for better planning and priotisation during development.
- the satisfaction of all the requriemtns will allow for evaluation of the integration into the software development lifecycle

\section{Functional Requirements}

\begin{table}[ht]
    \centering
    \small
    \begin{tabular*}{\textwidth}{@{\extracolsep{\fill}} p{0.2cm} p{2cm} p{7cm} p{4cm} @{}}
        \toprule
        \textbf{ID} & \textbf{Title} & \textbf{Description} & \textbf{Verification} \\
        \midrule
        F0 & Multi Trigger
        & The Pipeline can be triggered: manually, scheudled via cron, or by GitHub issue creation/labeling.
        & Runs can be found for these triggers \\[4pt]
        F1 & Issue Gathering
        & Retrieve GitHub repository issues and filter them for correct state and configured labeles \texttt{BUG}.
        & \texttt{gate} logs list of fetched issues.  \\[4pt]
        F2 & Code Checkout
        & Fetch the repository code into a fresh workspace and branch (via Docker mount).
        & After F1, workspace/ contains the correct source files. \\[4pt]
        F3 & Issue Localization
        & Use LLM to analyze the issue description and identify relevant files.
        & LLM output contains file paths with files that shall be edited. \\[4pt]
        F4 & Fix Generation
        & Use LLM to edit the identified files.
        & LLM output contains adjusted content for the identified files. \\[4pt]
        F5 & Change Validation
        & Run format, lint and relevant tests and capture pass/fail status.
        & Logs show build and test results. \\[4pt]
        F6 & Iterative Patch Generation (retry logic)
        & If F4 reports failures, retry F4--F5 up to X times.
        & After retries, either F4 passes or fails with no further retries. \\[4pt]
        F7 & Apply Patch
        & Commit LLM-generated edits and generate patch.
        & Git history shows a new commit which is referencing the Github issue \\[4pt]
        F8 & Result Reporting
        & Open a PR or post a comment on GitHub with the diff and summary metrics.
        & A PR or comment appears for each issue, showing diff and summary. \\[4pt]
        F9 & Logs and Metrics Collection
        & Provide log files and Metrics with fix-rate, attempt history, timings, token ussage.
        & A metrics file contains fields: issue, success, timings, stages. \\[4pt]

        % F1 & Per Issue Workflow
        % & Execute F1--F4 for each fetched issue in sequence and aggregate results.
        % & For N issues, produce N PRs (or "no-fix" comments) and N metric entries. \\[4pt]
        % F12 & GitHub Integration
        % & Use the GitHub API (with \texttt{GITHUB\_TOKEN}) to list issues, create branches, open PRs, and post comments.
        % & All GitHub API calls succeed without manual intervention. \\ [4pt]
        \bottomrule
    \end{tabular*}
    \caption{Functional requirements (F0--F8)}
\end{table}

\section{Non-Functional Requirements}

\begin{table}[ht]
    \centering
    \small
    \begin{tabular*}{\textwidth}{@{\extracolsep{\fill}} p{0.2cm} p{2cm} p{7cm} p{4cm} @{}}
        \toprule
        \textbf{ID} & \textbf{Title} & \textbf{Description} & \textbf{Verification} \\
        \midrule
        N1 & Containerized Execution
        & All agent code runs in CI runner in a Docker container to isloate it.
        & Workflow shows Docker container usage \\[4pt]
        N2 & Configurability
        & User can specify issue labels, branches, attempts, LLM models via YAML.
        & Changing the config file alters agent behavior accordingly. \\[4pt]
        N3 & Portability
        & The system can be deployed on any repository on GitHub.
        & ??? \\[4pt]
        N4 & Reproducibility
        & Runs are deterministic given identical repo state and config.
        & Multiple runs on the same issue report similar metrics. \\[4pt]
        N5 & Oberability
        & The system provides logs and metrics for each run, including execution times, token usage and success rate.
        & Logs and metrics files are generated after each run, containing all relevant data. \\[4pt]
        \bottomrule
    \end{tabular*}
    \caption{Non-Functional (N1--N5) requirements}
\end{table}
