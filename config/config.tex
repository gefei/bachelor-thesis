%	Pakete und Konfigurationen
%----------------------------------------------------------------------------------------
%\documentclass[twoside,twocolumn]{article}
\documentclass[oneside,bibliography=totocnumbered,BCOR=5mm]{scrbook}% Voreinstellungen entfernt.

\usepackage[latin1]{inputenc}
\usepackage{amsmath, amsthm, amssymb}
\usepackage[english]{babel} % Language hyphenation and typographical rules
\usepackage{marvosym}
\usepackage{graphicx}
\usepackage{csquotes}
\usepackage[hyphens]{url}
\usepackage[hidelinks]{hyperref}

\usepackage{xurl}
\usepackage{ragged2e}


%----------------------------------------------------------------------------------------
%	BIB.-Datei und Quellenverwaltung
%----------------------------------------------------------------------------------------
\usepackage[
  backend=biber, 
  style=numeric,
  sorting=none,
  uniquename=false,
  uniquelist=false,
  ibidtracker=false,
  language=auto,
  autolang=other,
  babel=other
  %dashed=false
  ]{biblatex}
\DeclareNameAlias{author}{family-given}
\DeclareNameAlias{editor}{family-given}

\DeclareDelimFormat{nameyeardelim}{\addcomma\space}
\DeclareFieldFormat{howpublished}{%
  \mkbibacro{URL}\addcolon\space
  \url{#1}%
}

% URLs im Literaturverzeichnis linksbündig umbrechen, Rest bleibt Blocksatz
\DeclareFieldFormat{url}{\begingroup\RaggedRight\url{#1}\endgroup}

% Diese Zeilen sorgen dafür, dass URLs leichter umbrochen werden
\setcounter{biburlnumpenalty}{9000} % Zahlen
\setcounter{biburlucpenalty}{9000}  % Großbuchstaben
\setcounter{biburllcpenalty}{9000}  % Kleinbuchstaben
\def\UrlBreaks{\do\/\do-\,\do=\do&\do?\do\#\do_}
\addbibresource{references/references.bib}
%\usepackage{natbib} % use natbib for references 
%----------------------------------------------------------------------------------------


\usepackage[sc]{mathpazo} % Use the Palatino font
\usepackage[T1]{fontenc} % Use 8-bit encoding that has 256 glyphs
\linespread{1.05} % Line spacing - Palatino needs more space between lines
\usepackage{microtype} % Slightly tweak font spacing for aesthetics

\usepackage[hmarginratio=1:1,top=32mm,columnsep=20pt]{geometry} % Document margins
\usepackage[hang, small,labelfont=bf,up,textfont=it,up]{caption} % Custom captions under/above floats in tables or figures
\usepackage{booktabs} % Horizontal rules in tables
\usepackage{lettrine} % The lettrine is the first enlarged letter at the beginning of the text
\usepackage{enumitem} % Customized lists
\setlist[itemize]{noitemsep} % Make itemize lists more compact

\urlstyle{same} % Use same font as text

%\usepackage{abstract} % Allows abstract customization
%\renewcommand{\abstractnamefont}{\normalfont\bfseries} % Set the "Abstract" text to bold
%\renewcommand{\abstracttextfont}{\normalfont\small\itshape} % Set the abstract itself to small italic text

% \usepackage{titlesec} % Allows customization of titles
%\renewcommand\thesection{\Roman{section}} % Roman numerals for the sections
%\renewcommand\thesubsection{\roman{subsection}} % roman numerals for subsections
%\titleformat{\section}[block]{\large\scshape\centering}{\thesection.}{1em}{} % Change the look of the section titles
%\titleformat{\subsection}[block]{\large}{\thesubsection.}{1em}{} % Change the look of the section titles

%\usepackage{fancyhdr} % Headers and footers
%\pagestyle{fancy} % All pages have headers and footers
%\fancyhead{} % Blank out the default header
%\fancyfoot{} % Blank out the default footer
%\fancyhead[C]{Ethics in Progress (EiP) $\bullet$ 2019 } % Custom header text
%\fancyfoot[RO,LE]{\thepage} % Custom footer text

\usepackage{titling} % Customizing the title section

%----------------------------------------------------------------------------------------
%	Listings
%----------------------------------------------------------------------------------------
\usepackage{listings}
\usepackage{color}
\usepackage{xcolor}

\definecolor{mygreen}{rgb}{0,0.6,0}
\definecolor{mygray}{rgb}{0.5,0.5,0.5}
\definecolor{mymauve}{rgb}{0.58,0,0.82}
\definecolor{lightgray}{rgb}{0.95,0.95,0.95}
\definecolor{darkblue}{rgb}{0.0,0.0,0.6}

% Default listing style
\lstset{
  backgroundcolor=\color{white},
  basicstyle=\footnotesize\ttfamily,
  breakatwhitespace=false,
  breaklines=true,
  captionpos=b,
  frame=single,
  keepspaces=true,
  numbers=left,
  numbersep=5pt,
  numberstyle=\tiny\color{mygray},
  rulecolor=\color{black},
  showspaces=false,
  showstringspaces=false,
  showtabs=false,
  stepnumber=1,
  tabsize=2,
  title=\lstname
}

% Python code style
\lstdefinestyle{python}{
  language=Python,
  backgroundcolor=\color{lightgray},
  basicstyle=\footnotesize\ttfamily,
  commentstyle=\color{mygreen},
  keywordstyle=\color{darkblue}\bfseries,
  stringstyle=\color{mymauve},
  frame=single,
  numbers=left,
  breaklines=true
}

% JSON style
\lstdefinestyle{json}{
  backgroundcolor=\color{white},
  basicstyle=\footnotesize\ttfamily,
  frame=single,
  numbers=left,
  breaklines=true,
  string=[s]{"}{"},
  stringstyle=\color{mymauve},
  comment=[l]{//},
  commentstyle=\color{mygreen}
}

% Log style (no syntax highlighting, clean appearance)
\lstdefinestyle{log}{
  backgroundcolor=\color{lightgray},
  basicstyle=\footnotesize\ttfamily,
  frame=single,
  numbers=none,
  breaklines=true,
  showstringspaces=false,
  columns=fullflexible
}

% Command/terminal style
\lstdefinestyle{terminal}{
  backgroundcolor=\color{black},
  basicstyle=\footnotesize\ttfamily\color{white},
  frame=single,
  numbers=none,
  breaklines=true,
  showstringspaces=false
}


\usepackage{parskip} % To avoid indentation of paragraphs and use line breaks instead

\usepackage{placeins}

\usepackage{float}

\usepackage{longtable} %tables across multiple pages

\usepackage{acronym}

\usepackage{pdfpages}

\usepackage{ragged2e}

\usepackage{tabularx}


\DeclareFieldFormat{url}{\begingroup\RaggedRight\url{#1}\endgroup}

% Diese Zeilen sorgen dafür, dass URLs leichter umbrochen werden
\setcounter{biburlnumpenalty}{9000} % Zahlen
\setcounter{biburlucpenalty}{9000}  % Großbuchstaben
\setcounter{biburllcpenalty}{9000}  % Kleinbuchstaben


\RedeclareSectionCommand[
  beforeskip=10pt,  % Space before chapter title (default is about 50pt)
  afterskip=30pt   % Space after chapter title
]{chapter}

\hypersetup{
  pdfauthor={Justin Gebert},
  pdftitle={Integrating LLM based Automated Bug Fixing into Continuous Integration - Analysis of Potentials and Limitations},
  pdfsubject={Bachelor Thesis},
  pdfkeywords={AI, Automated Program Repair, Continuous Integration, Large Language Models},
  pdfcreator={Justin Gebert},
  pdfproducer={Justin Gebert}
}

\usepackage{chngcntr}
\counterwithout{footnote}{chapter}